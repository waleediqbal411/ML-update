\section{Related Work}
\label{sec:relatedwork}

In this section, we present related work and highlight the novelty of this article. Bibliometrics is an established field in which the major trends of research fields are studied rigorously. A number of bibliometrics studies have been conducted in various fields to gain useful insights through the analysis of authorship and publication trends of different research outlets and areas  \citep{nobre2017scientific,fernandes2017evolution,serenko2009scientometric,chiu2010publish, rajendran2011scientometric,nattar2009indian,yin2017dancing,iqbal2019bibliometric,iqbal2019five}. These bibliometric analyses are not confined to the authorship based meta-data analysis of venues. 

Some authors have also undertaken quantitative analysis on the top ACM conferences. The purpose of these studies is to determine the genre of the article and to understand the publication culture of these conferences \citep{flittner2018survey}. These related studies do not explain which factors of the article affect the productivity parameters and the information about the correlation between important parameters required to analyze the productivity of different entities. Many previous works have performed an analysis on the content of various research areas using topic modeling \citep{paul2009topic} and keyword-based analysis \citep{choi2011analysis}.

A number of studies have used social networking analysis for social sciences and medical science research to find the most significant collaborating entities \citep{savic2017analysis,wagner2017growth,didegah2018co,borgatti2009network,waheed2018bibliometric}, using social network analysis on generally social media data and altmetric data \citep{hassan2017measuring}. Social media analysis has not been used to determine the communities in machine learning research due to which we do not yet have complete insights into the collaborating patterns that exist in machine learning research. 

Limited work has focused on using bibliometric or scientometric techniques to analyze the publication on the field of machine learning.  Our work is also different in that we perform a detailed bibliometric analysis on the machine learning literature including an analysis of the effects of various features of the article (such as the graphical and mathematical elements and the numbers of references) on the article's productivity metrics as defined in the field of bibliometrics.


Bibliometric analyses can also be utilized to see the extent of the incorporation of related research. Reference count in a article is the simplest way to observe the inclusion of related research and literature review. Different researchers analyzed referencing patterns in research articles to identify incorporation of the latest studies relating to a research article \citep{heilig2014scientometric} and citation analysis of the productivity of various research entities \citep{hamadicharef2012scientometric,bartneck2009scientometric}. These studies do not explain how the references are affected by the type of article venue. Some researchers have also studied the overall publications behaviour in machine learning and provide some advice to reviewers of machine learning  papers but they are not covering geographical aspects of machine learning research \citep{lipton2018troubling}. To the best of our knowledge, no such study has been undertaken in machine learning until now.
