\section{Introduction}

Bibliometric analysis of a literature is a crucially important source of objective knowledge and information about the quantity and quality of scientific work \citep{narin1994bibliometrics}. In this work we perform a bibliometric analysis of the the literature of the field of machine learning. This breadth-wise knowledge saves ample amount of time for researchers to get started with the research of a domain and helps inform about the major trends observed in machine learning publications.

Research literature of Machine learning is being generated and evolved on a very rapid pace. This trend is not only true for machine learning based studies but overall computer science based studies as well. Everyday, a large number of new authors are entering their contributions to the research literature of machine learning. Arxiv, a popular paper preprint hosting service, has also facilitated greatly these new authors in propagation of their research work.

There are several article genres in machine learning, such as conference articles, letters, editorials, and empirical studies. To keep the scope of this study to manageable proportions, we have focused on four major venues in this area. We have selected four exemplar venues that represent the highest standard of research in the field of machine learning---namely, International Conference on Machine Learning (ICML), Journal of Machine Learning Research (JMLR), Conference on Neural Information Processing Systems (NeurIPS), and SIGKDD Conference on Knowledge Discovery and Data Mining (SIGKDD). ICML and JMLR deals with machine learning based studies, NeurIPS deals with neural networks based studies whereas SIGKDD publishes data mining based studies. 
% \gareth{COuld we add some justification here, e.g. look up CORE rankings, and say we get the top $n$ venues?}

Towards this end, we statistically analyze 16 years of accepted articles published in the four major machine learning venues (ICML, JMLR, NeurIPS, and SIGKDD), explore various bibliometric questions, and examine the publication behaviors of several research entities and how these are affected by the elements of articles. We also analyze popular topics in periodicals on machine learning and the effects of several parameters on the citations of an article. We believe that a deep study of the articles published in these venues can not only provide insight into current publication practice, but can also inform about the temporal evolution of the publishing trends in these venues. 

We structure our work around two major comparisons. First, we directly compare publication trends in all four venues, to understand how these distinct publication types differ. Second, we compare trends over time to understand how they have evolved. 

Our aim is to investigate changes in publication behavior and collaboration patterns of distinctive authors, institutes and countries in the various machine learning publications. Our goal is therefore to provide generalized insights into the publication trends in the field of machine learning. We also aim to answers questions such as the following: Which topics are popular in which regions of the world? What are the topics discussed by the top authors in their articles in the various publications? Which parameters affect the citations of an article? 

The \textit{key contribution of this article} is to develop a methodology and framework for performing a comprehensive bibliometric analysis on machine learning research and the public release of a comprehensive dataset. To facilitate future research in this area, we have publicly released our dataset including metadata, content, and citation related data for the articles published in ICML, JMLR, NeurIPS, and SIGKDD from 2004 to 2019\footnote{link to be shared}.   

The rest of this article is structured as follows. In section \ref{sec:relatedwork}, we discuss related previous research work. The bulk of our investigations focus on the publication trends in machine learning publications in all four venues (Sections \ref{sec:methodology}--\ref{sec:citation}). In Section \ref{sec:methodology}, our dataset is described and our methodology is broadly outlined. A detailed bibliographic focused on comparison of all four venues is presented in Sections \ref{sec:metadata}, \ref{sec:content}, \ref{sec:citation} in which metadata analyses, content-based analyses, citation-based analyses are presented respectively. The paper is finally concluded in Section \ref{sec:conclusion}.

